\documentclass[12pt]{report}
\usepackage{amsmath, amssymb, amsthm}
\usepackage{graphicx}
\usepackage{hyperref}
\usepackage{cite}
\usepackage{physics}
\usepackage{algorithm}
\usepackage{algorithmic}
\usepackage{tikz}
\usepackage{caption}
\usepackage{subcaption}
\usepackage{setspace}

\newcommand{\opnorm}[1]{\left\lVert #1 \right\rVert}


\usepackage[a4paper, left=1.5cm, right=1.5cm, top=2cm, bottom=2cm]{geometry}

% Formatting
\setlength{\parskip}{0.5em}
\setlength{\parindent}{0pt}
\singlespacing

\title{Quantum Walks and Application to Quantum Money}
\author{Ali Mousavi}
\date{April 2025}







\begin{document}




\maketitle
\tableofcontents


\newpage
\textbf{Abstract}

This thesis explores the foundations of quantum computation, focusing on quantum walks and their application to quantum money. Quantum walks, particularly continuous-time quantum walks based on group actions, serve as a powerful computational tool with applications in search algorithms and cryptographic protocols. We examine their mathematical structure and their advantages over classical random walks, emphasizing their efficiency in state evolution and probability distribution spreading. As a part of this work, we examine efficient implementations of transforms such as the Quantum Fourier Transform (QFT) and the Quantum Hartley Transform (QHT), analyzing their role in encoding quantum states for secure cryptographic applications. In particular, we discuss a novel instantiation of a quantum money scheme based on QHT, leveraging its unique properties for improved security and efficiency.
To ensure the robustness of this quantum money scheme, we develop a verification mechanism utilizing quantum walks. Unlike previous approaches, which rely on standard quantum state measurements, our method employs continuous-time quantum walks to authenticate quantum money, preventing counterfeiting while maintaining computational feasibility. Additionally, we present a detailed discussion on the efficient implementation of this scheme, including optimized circuit designs and error mitigation strategies.








\chapter{Introduction}
Quantum money, first introduced in a seminal paper by Wiesner [29], is a form of currency represented by quantum states. Unlike classical money, an ideal quantum bill cannot be counterfeited due to the no-cloning theorem of quantum mechanics, which prohibits making an identical copy of an unknown quantum state. Wiesner’s original scheme (now termed a \emph{private-key quantum money} scheme) had significant practical drawbacks: it required the issuing bank’s secret key to verify each banknote, meaning the bank had to be involved in every transaction. This reliance on the bank for verification severely limits the usability of private-key quantum money schemes.

In 2009, Aaronson [1] proposed the first \emph{public-key quantum money} scheme, in which anyone can verify a quantum banknote while only the bank can mint (issue) new valid banknotes. Public-key quantum money removes the transactional bottleneck of involving the bank every time. Aaronson’s specific scheme was later broken by Lutomirski et al.~[24]. In the years since, several alternative public-key quantum money constructions have been explored [2, 17, 30, 19, 20, 23, 31]. However, each of these proposals has either been broken by further cryptanalysis [14, 26, 7, 23] or relies on unconventional cryptographic assumptions, making their security or practicality questionable.

\medskip\noindent\textbf{Quantum Money from Group Actions and the Fourier Transform.} 
A promising candidate for secure public-key quantum money based on more standard assumptions was recently proposed by Zhandry [31]. In Zhandry’s scheme, the core idea is to use an abelian group action as the foundation for the quantum state. Each money state (quantum banknote) is prepared as a \emph{group-action Fourier state}, and the corresponding serial number is an element of the underlying group (we will elaborate on these terms in later chapters). The verification algorithm in this scheme uses a group-action phase kickback routine to extract the serial number from the quantum banknote and check its validity. Doliskani [15] later proved that Zhandry’s scheme is secure in the generic group action model, providing evidence for its security under well-defined assumptions.

\medskip\noindent\textbf{This Work – Motivation and Contributions.} 
In this thesis, we adapt Zhandry’s group-action quantum money scheme by replacing its use of the Quantum Fourier Transform with the \emph{Quantum Hartley Transform} (QHT) over finite abelian groups. The motivation behind this substitution is multifold. First, using the Hartley transform causes the banknotes to have \emph{real} amplitudes instead of complex amplitudes. We believe this shift from complex to real quantum states may offer both computational and theoretical advantages. For example, certain mathematical identities hold for any two real orthonormal bases that do not generally hold for complex bases; such differences can affect the properties of quantum states and were a barrier in prior analyses (indeed, an attempted “quantum lightning” construction in [31] failed due to properties of complex phases that do not arise with real amplitudes). Beyond this specific context, our work is a first step toward employing real-valued quantum transforms in quantum cryptography and quantum algorithms. To the best of our knowledge, this is the first significant quantum cryptographic construction that makes essential use of the quantum Hartley transform. We hope that exploring real-amplitude quantum states will inspire further research into new quantum algorithms and optimizations for real-valued transforms.

We now summarize our main contributions:
\begin{itemize}
    \item We propose a new \textbf{public-key quantum money scheme} based on abelian group actions and the quantum Hartley transform. This scheme is an adaptation of Zhandry’s group-action money scheme, modified to output real-amplitude quantum states. 
    \item When using the Hartley transform instead of the Fourier transform, the original verification method from Zhandry’s scheme is no longer straightforward and requires the use of twists to function correctly. Instead, we offer a more direct verification approach based on quantum walks.
    \item We show that the serial number associated with a money state (which is a hidden group element) can be efficiently computed from the quantum state. In particular, we develop a \textbf{continuous-time quantum walk algorithm} to extract the serial number of a given banknote state. This algorithm leverages the structure of the group action and the spectral properties of an associated Cayley graph. Our method demonstrates that one can recover the hidden group element (serial) with high success probability, which not only underpins the new verification technique but is also of independent interest.
    \item In the course of our construction, we introduce a new algorithm for efficiently implementing the \textbf{quantum Hartley transform} and related real transforms. We present a recursive technique that exploits the structure of the Hartley transform, yielding a lower gate complexity than prior approaches in the literature.
\end{itemize}

In summary, our work combines ideas from quantum cryptography (public-key quantum money), quantum walks, and quantum signal processing (Hartley and related transforms) to create a novel and potentially practical quantum money scheme. 

\medskip\noindent\textbf{Thesis Organization.} 
The remainder of this thesis is organized as follows. In **Chapter 2**, we provide background on the basics of quantum computation and the theory of group actions, including the concept of cryptographic group actions which underlie our money scheme. **Chapter 3** covers quantum walks, both continuous-time and discrete-time, outlining their differences and importance in quantum algorithms. In **Chapter 4**, we apply the quantum walk framework to group actions: we describe how a continuous-time quantum walk on a group action can be simulated efficiently, a crucial step for our serial number extraction algorithm. **Chapter 5** details the quantum money scheme itself and the role of the Hartley transform in its construction. We explain how the scheme is formulated and discuss the advantages and challenges introduced by using the Hartley transform. Finally, **Chapter 6** presents the verification algorithm for the Hartley-based quantum money. This chapter shows how we use quantum walks (as developed in Chapter 4) to reliably verify banknotes and compute their serial numbers, thereby addressing the verification issues that arose from the use of real amplitudes.
\vspace{1em}



\chapter{Background of Quantum Computation and Group Actions}



\section{Quantum Computation Basics}

Quantum computation operates on information stored in quantum states. The fundamental unit of quantum information is the \emph{qubit}, which, unlike a classical bit, can exist in a superposition of basis states. A qubit is described by a state $|\psi\rangle = \alpha |0\rangle + \beta |1\rangle$ (in Dirac notation), where $\alpha,\beta \in \mathbb{C}$ and $|\alpha|^2 + |\beta|^2 = 1$. A system of $n$ qubits resides in a $2^n$-dimensional Hilbert space and its state can be a superposition of all $2^n$ basis states $|b_1b_2\cdots b_n\rangle$. 

Computation is carried out by unitary transformations on these states. A quantum algorithm is typically a sequence of quantum gates (unitary operators) applied to an initial state (usually a simple basis state like $|00\cdots0\rangle$), followed by a measurement of some qubits to obtain a classical result. A crucial aspect of quantum computation is that certain transformations—such as the Fourier transform over an $N$-element group—can be implemented very efficiently on a quantum computer (in $\operatorname{poly}(\log N)$ time), whereas their classical counterparts might require $\operatorname{poly}(N)$ time.

\subsection{The Quantum Fourier Transform}

One example important to this work is the \textbf{quantum Fourier transform} (QFT). For an abelian group $G$ of order $N$, the QFT is the unitary map
\[
|g\rangle \mapsto \frac{1}{\sqrt{N}}\sum_{h \in G} \chi_h(g)\, |h\rangle,
\]
where $\{\chi_h\}$ are the characters (complex exponential homomorphisms) of $G$. When $G = \mathbb{Z}_N$ (the integers mod $N$), the QFT takes $|j\rangle$ to $\frac{1}{\sqrt{N}}\sum_{k=0}^{N-1} e^{2\pi i j k / N} |k\rangle$. 

Let $G \cong \mathbb{Z}_{N_1} \oplus \cdots \oplus \mathbb{Z}_{N_k}$ be a decomposition of $G$. The character $\chi(a, x)$ is then given by
\[
\chi(a, x) = \omega_{N_1}^{a_1x_1} \cdots \omega_{N_k}^{a_kx_k}, \quad \text{where } \omega_M = \exp(2\pi i/ M).
\]

The Fourier transform of a function $f: G \to \mathbb{C}$ is given by
\[
\hat{f}(a) = \frac{1}{\sqrt{|G|}} \sum_{x \in G} \chi(a, x) f(x),
\]
and the QFT of a quantum state $\sum_{g \in G} f(g) \ket{g}$ is $\sum_{x \in G} \hat{f}(x) \ket{x}$.

\paragraph{Fourier Basis and Group Actions.} For a regular group action $(G, X, *)$, define
\[
\ket{S^{(h)} * y} = \frac{1}{\sqrt{|S|}} \sum_{g \in S} \chi(g, h) \ket{g * y}.
\]
For fixed $x \in X$, we obtain two orthonormal bases of $\mathbb{C}^X$: the standard basis $\{ \ket{x} : x \in X \}$, and the Fourier basis
\[
\ket{G^{(h)} * x} = \frac{1}{\sqrt{|G|}} \sum_{g \in G} \chi(g, h) \ket{g * x}.
\]
These states are eigenstates of the action unitary $U_k: \ket{y} \mapsto \ket{k * y}$:
\[
U_k \ket{G^{(h)} * x} = \chi(-k, h) \ket{G^{(h)} * x}.
\]

\paragraph{The cmpIndex Algorithm.}
Given a state $\ket{G^{(h)} * x}$, one can recover $h$ using phase kickback. Starting from $\ket{G^{(h)} * x} \ket{0}$, apply:
\begin{itemize}
    \item QFT to second register,
    \item Controlled unitary $\sum_{k \in G} U_k \otimes \ket{k}\bra{k}$,
    \item Inverse QFT to second register.
\end{itemize}
This yields $\ket{G^{(h)} * x} \ket{h}$, extracting $h$ efficiently.

\subsection{The Quantum Hartley Transform}

Another concept we use is the \textbf{quantum Hartley transform} (QHT), which replaces complex exponentials with real sinusoidal kernels.

\paragraph{Definition.}
For $G = \mathbb{Z}_N$, define
\[
\operatorname{QHT}_N: \ket{a} \mapsto \frac{1}{\sqrt{N}} \sum_{y = 0}^{N - 1} cas\left(\frac{2\pi a y}{N}\right) \ket{y}, \quad \text{where } cas(x) = \cos(x) + \sin(x).
\]

For general abelian $G = \mathbb{Z}_{N_1} \oplus \cdots \oplus \mathbb{Z}_{N_k}$, the Hartley transform of $f : G \to \mathbb{R}$ is
\begin{equation}
    \operatorname{QHT}_G(f)({a}) = \frac{1}{\sqrt{|G|}} \sum_{{y} \in G} cas(2\pi \langle {a}, \alpha({y}) \rangle) f({y}),
\end{equation}
where $\alpha({y}) = y_1 / N_1 + \cdots + y_k / N_k$.

\paragraph{Unitary Equivalence.}
Using the identity
\[
cas(2\pi \langle {a}, \alpha({y}) \rangle) = \frac{1 - i}{2}\chi({a}, {y}) + \frac{1 + i}{2}\chi(-{a}, {y}),
\]
we obtain:
\begin{equation}
\operatorname{QHT}_G = \frac{1 - i}{2} \operatorname{QFT}_G + \frac{1 + i}{2} \operatorname{QFT}_G^*,
\end{equation}
showing that $\operatorname{QHT}_G^2 = {1}$ and hence the QHT is unitary.

\paragraph{Quantum Formulation.}
The quantum Hartley transform over $G$ maps:
\[
\sum_{{x} \in G} f({x}) \ket{{x}} \mapsto \sum_{{a} \in G} \operatorname{QHT}_G(f)({a}) \ket{{a}}.
\]
Because the cas function is real-valued, the resulting amplitudes remain real—a fact that plays a crucial role in real-amplitude quantum protocols.


\section{Group Actions in Cryptography}
At the heart of our quantum money scheme is the notion of a group action. A \textbf{group action} of a group $G$ on a set $X$ is a function $*\colon G \times X \to X$ that satisfies two properties: (1) the identity element $e \in G$ acts trivially on every $x \in X$ ($e * x = x$), and (2) the action is compatible with the group operation, meaning $g*(h* x) = (gh)*x$ for all $g,h \in G$ and $x \in X$. We will often denote the result of $g$ acting on $x$ by $g*x$ or simply $g x$ when the action is clear from context.

The orbit of an element $x \in X$ under the action is the set $G * x = \{g*x : g \in G\} \subseteq X$. If there is exactly one orbit (i.e., $G * x = X$ for some $x$), the action is called \emph{transitive}. If, in addition, no non-identity group element fixes any $x$ (i.e., $g * x = x$ implies $g = e$), then the action is \emph{free}. A group action that is both transitive and free is sometimes called a \textbf{regular action}. In a regular action, for any two elements $y, z \in X$, there is a unique group element $g$ such that $g * y = z$. In other words, $G$ acts like a permutation group that moves elements of $X$ around, and knowing the starting and ending point uniquely determines the “move” $g$. In this case $|X| = |G|$, and we can think of $X$ as essentially a copy of $G$ with the group structure “hidden” by the action.

Group actions are ubiquitous in modern cryptography. A \textbf{cryptographic group action} is one where the action is easy to compute in the forward direction, but inverting the action (recovering $g$ from $x$ and $g*x$) is computationally hard. This essentially functions as a one-way function: given $x$ and $g$, it is easy to compute $y = g*x$, but given $x$ and $y$ it is infeasible to find $g$ (except with negligible probability). One classical example is the discrete logarithm problem: consider the cyclic group $G = \mathbb{Z}_p^*$ (the multiplicative group of a prime field) acting on itself by exponentiation. Let $X = G$ and define $g * x = x^g \pmod p$ (treating group elements as integers). Here $g$ is an exponent. This is a group action (in fact, just the group’s own operation written differently), and computing $x^g \bmod p$ is efficient. However, given $x$ and $y = x^g$, finding $g$ requires solving a discrete log problem, which is believed to be hard for suitable choices of $p$. Thus, exponentiation can be viewed as a one-way group action.

Another prominent example comes from elliptic-curve cryptography: the action of an ideal class group on a set of elliptic curves. In schemes like CSIDH (Commutative Supersingular Isogeny Diffie–Hellman), an abelian group of ideal classes $G$ acts on the set $X$ of supersingular elliptic curves by isogeny (an isogeny is a structure-preserving map between curves). Starting from a curve $x_0$, one can efficiently compute $g * x_0$ (which is another elliptic curve) for any group element $g$, but given two curves $x_0$ and $x = g * x_0$, it is believed to be computationally hard to recover $g$. This hard problem is the foundation of isogeny-based cryptography. We refer to such $G$ and $X$ as a \emph{cryptographic group action} pair.

Cryptographic group actions provide a natural platform for public-key quantum money schemes. In such schemes, the secret “serial number” of a quantum banknote can be an element $g \in G$, and the quantum state of the banknote can be some function of $g*x_0$ for a public base point $x_0 \in X$. The hardness of inverting the action ($x_0$ and $x = g*x_0$ do not easily reveal $g$) contributes to the security against counterfeiting, while the structure of the group action can be used to efficiently verify authenticity (as we will see with the Fourier or Hartley transform techniques).

We will assume throughout that we have a family of abelian group actions $(G,X)$ that are cryptographically suitable: $|G|$ is large (growing with the security parameter), the action is efficiently computable, transitive and free (so $|X|=|G|$), and inverting the action is infeasible without the secret key. Such families are conjectured to exist (for example, the isogeny-based actions, and others discussed in cryptographic literature). 

Finally, we note that working in the \textbf{generic group model} (or generic group \emph{action} model) is a common theoretical approach to analyze security. In a generic model, the group elements are treated as black-box labels without exploiting any special algebraic structure beyond group axioms. Doliskani’s proof of security for Zhandry’s scheme [15] was in such a generic model for group actions, lending confidence that no “generic” attack can break the scheme. In this thesis, we focus on designing the scheme and algorithms at a high level and assume the underlying group action is secure; a detailed security proof is beyond our scope, though we rely on prior results for reassurance.





\chapter{Quantum Walks -- Continuous and Discrete}

Quantum walks are the quantum-mechanical counterparts of classical random walks on graphs. Instead of a probability distribution over vertices, a quantum walk evolves a superposition of vertex states. Interference between multiple paths leads to markedly different behavior from classical walks, often yielding faster spreading and algorithmic speedups~\cite{physics_stackexchange}.

In particular, both major models of quantum walks --- continuous-time and discrete-time --- achieve \emph{ballistic spreading} (standard deviation growing linearly in time) in contrast to the diffusive $O(\sqrt{t})$ spread of classical random walks~\cite{childs2003quantum}. Quantum walks have become a central framework in quantum computing, underpinning various algorithms and even universal quantum computation~\cite{file_gwrqpr3gkui6gav6sze4dj}.

In this chapter, we introduce continuous- and discrete-time quantum walks, compare their properties, and discuss their implementation on highly symmetric graphs known as Cayley graphs. We then explore how group actions and symmetry can be leveraged to analyze and construct quantum walks on these graphs.



\section{Continuous-Time Quantum Walk}


A \textbf{continuous-time quantum walk (CTQW)} is defined by a time-dependent Schrödinger equation on the vertices of a graph. Given a graph $G=(V,E)$ with $\abs{V}=N$ vertices, one assigns a basis state $\ket{j}$ to each vertex $j \in V$. The walk is governed by a Hermitian Hamiltonian $H$ that respects the graph's structure. A common choice is to take $H$ proportional to either the adjacency matrix $A$ of $G$ or the graph Laplacian $L$. For example, one may set
\[
H = \gamma A \quad \text{or} \quad H = \gamma L,
\]
where $\gamma$ is a constant coupling rate with dimensions of (time)$^{-1}$.

The Hamiltonian connects adjacent vertices: $\bra{j}H\ket{k} \neq 0$ if and only if $(j,k) \in E$. The state of the walk $\ket{\psi(t)} = \sum_{j \in V} \psi_j(t) \ket{j}$ evolves according to the Schrödinger equation:
\[
i \frac{d}{dt} \ket{\psi(t)} = H \ket{\psi(t)},
\]
with formal solution
\[
\ket{\psi(t)} = e^{-iHt} \ket{\psi(0)}.
\]

In coordinates, if we use the Laplacian $L$ defined by $L_{jk} = -\deg(j)\delta_{jk} + A_{jk}$, the equation becomes
\[
i \frac{d}{dt} \psi_j(t) = \sum_{k \in V} L_{jk} \psi_k(t),
\]
which mirrors the equation of a classical continuous-time random walk except for the factor $i$. Since $L$ (and $A$) are Hermitian, the quantum dynamics are unitary and conserve total probability (i.e., the norm of the state vector), in contrast to the irreversible nature of classical diffusion.

The continuous-time quantum walk model was first formalized by Farhi and Gutmann in 1998 as a quantum analogue of continuous Markov processes.

\subsection{Hamiltonian Choices}

One can equivalently define a CTQW using $H = A$ (the adjacency matrix) instead of the Laplacian. For an undirected $d$-regular graph, $L = dI - A$, and using $H=A$ introduces only an overall phase factor $e^{-i d t}$ in the evolution, since $dI$ contributes a global phase. Thus, on regular graphs, the choice of Laplacian versus adjacency matrix is unimportant up to a global phase. In general, the adjacency matrix is often used because it directly encodes graph connectivity. Farhi and Gutmann’s original construction corresponds to $H=A$.

Throughout this thesis, we primarily consider $H = A$ for continuous-time walks on unweighted graphs.

\subsection{Example – CTQW on a Hypercube}

The $n$-dimensional Boolean hypercube $Q_n$ is the graph on $2^n$ vertices $\{0,1\}^n$, where edges connect vertices differing in one bit. This is a $d = n$ regular graph, and can be viewed as the Cayley graph of $(\mathbb{Z}_2)^n$ with the generating set consisting of flipping each bit.

The adjacency matrix of $Q_n$ can be written as a sum of $n$ commuting terms, each acting on one bit:
\[
A_{Q_n} = \sum_{j=1}^{n} \sigma_x^{(j)},
\]
where $\sigma_x^{(j)}$ is the Pauli $X$ operator acting on the $j$-th bit (and identity on the others). Because all $\sigma_x^{(j)}$ commute, the time-evolution operator factorizes:
\[
e^{-i A_{Q_n} t} = \prod_{j=1}^{n} e^{-i \sigma_x^{(j)} t} = \bigotimes_{j=1}^{n} \left( \cos t \, I - i \sin t \, \sigma_x^{(j)} \right),
\]
yielding an analytical solution for the walk. In particular, starting from the state $\ket{0^n}$, the walker's state remains an even superposition over all vertices at Hamming distance $k$ at time $t$, with amplitudes given by the $k$-th term in the expansion of $(\cos t - i \sin t)^n$.

This shows that the quantum walk spreads across the hypercube exponentially faster than a classical random walk, which requires $O(n \log n)$ steps to approach a uniform distribution. The hypercube example illustrates how symmetry (in this case, the direct product structure) enables explicit solutions.

\subsection{Algorithmic Applications}

Continuous-time quantum walks have been employed to achieve algorithmic speedups. A famous example is the \emph{glued trees traversal problem} by Childs et al. (2003), where a CTQW on a graph formed by two randomly "glued" binary trees can find the unique exit vertex exponentially faster than any classical algorithm. This result gave the first exponential separation between quantum and classical query complexity in a black-box setting.

Another application is \emph{spatial search} on graphs. Childs and Goldstone (2004) showed that a CTQW can find a marked vertex on a $d$-dimensional lattice faster than classical diffusion. In the limit $d \to \infty$, the CTQW dynamics converge to the continuous Dirac equation.

More generally, CTQWs provide a natural framework for quantum search algorithms on graphs, often achieving Grover-like speedups or better. It is also known that continuous-time quantum walks are \textbf{universal for quantum computation}, meaning that sufficiently complex graphs can encode arbitrary quantum logic operations.



\section{Discrete-Time Quantum Walks}



In a \textbf{discrete-time quantum walk (DTQW)}, evolution proceeds in discrete time steps. Formally, one defines a unitary step operator $U$ that is repeatedly applied to an initial state. Designing such a $U$ for an arbitrary graph is nontrivial, since a naive attempt to “jump to a random neighbor” quantumly is not unitary.

The resolution is to enlarge the Hilbert space by introducing an auxiliary \emph{coin register} that provides additional degrees of freedom for unitary evolution. This model, first proposed by Watrous in the context of quantum graph algorithms, leads to the \emph{coined discrete-time quantum walk}.

\subsection{Coined Quantum Walk}

For a graph $G=(V,E)$, the state space is $\mathcal{H} = \mathbb{C}^{|V|} \otimes \mathbb{C}^d$, where $d$ is the maximum vertex degree (or a fixed value in regular graphs). A basis state $\ket{j,k}$ represents the walker at vertex $j$ with coin state $k$, where the coin index can be thought of as a label for one of $j$'s outgoing edges.

Each step of the walk is a composition of two operations:
\begin{enumerate}
    \item A \emph{coin flip} $C$ acting on the coin register (possibly conditioned on the current vertex),
    \item A \emph{shift} $S$ that moves the walker to a neighboring vertex determined by the coin state.
\end{enumerate}

In the case of a $d$-regular graph, a common choice is the Grover diffusion operator as the coin:
\[
C = \sum_{j \in V} \ket{j}\bra{j} \otimes C_j, \quad C_j := 2 \ket{\chi_j}\bra{\chi_j} - I_d,
\]
where $\ket{\chi_j} := \frac{1}{\sqrt{\deg(j)}} \sum_{k : (j,k) \in E} \ket{k}$ is the uniform superposition over the coin labels of neighbors of $j$. This is a Householder reflection about $\ket{\chi_j}$ and acts as a Grover diffusion operator on the coin space.

Next, the shift operator $S$ acts as
\[
S \ket{j,k} = \ket{k,j}, \quad \text{for each } (j,k) \in E,
\]
i.e., if the coin state $k$ indicates a neighbor of $j$, the shift $S$ moves the walker to vertex $k$ and updates the coin to $j$. The overall unitary step of the walk is then:
\[
U = S C.
\]

By construction, $U$ is unitary and effects a “quantum hop” on the graph at each time step. On a regular graph, $C$ performs a uniform mixing over possible directions (like tossing a $d$-sided quantum coin), and $S$ implements a conditional shift. On irregular graphs, care must be taken to define appropriate coin spaces, or one may instead use Szegedy’s alternate framework.

\subsection{Szegedy’s Walk}

In 2004, Szegedy proposed a different formulation of DTQW based on classical Markov chains. Here, no coin register is needed. Instead, the walk is defined on $\mathbb{C}^N \otimes \mathbb{C}^N$, with basis states $\ket{j,k}$ corresponding to directed edges.

Given a classical transition matrix $P$ (an $N \times N$ stochastic matrix), one defines:
\[
\ket{\psi_j} := \ket{j} \otimes \sum_k \sqrt{P_{kj}} \ket{k},
\quad
\Pi := \sum_j \ket{\psi_j}\bra{\psi_j}.
\]
Then the walk step is:
\[
U = S (2\Pi - I),
\]
where $S$ again swaps the two registers: $S\ket{j,k} = \ket{k,j}$.

This construction yields a unitary $U$, and when $P_{kj} = \frac{1}{\deg(j)} A_{kj}$ (i.e., a symmetric random walk), Szegedy’s walk is equivalent to the coined model.

Szegedy’s approach is especially useful for quantizing non-regular Markov chains and is widely used in algorithm design—for example, in the element distinctness problem.

\subsection{Properties and Applications}

Discrete-time quantum walks share many features with their continuous-time counterparts. On the infinite line, the standard coined walk (with Hadamard coin) spreads ballistically, creating a bimodal distribution, unlike the diffusive spread of a classical random walk.

Both DTQW and CTQW have been used to achieve quadratic speedups in search problems. In many cases, the two models can simulate each other. DTQWs can even achieve \textbf{universal quantum computation}: Childs et al. (2009) constructed a coined DTQW that simulates an arbitrary quantum circuit.

In practice, the coined model is convenient for quantum circuit implementations, as the step $U = SC$ can be decomposed into standard quantum gates. Experimental DTQW implementations have been realized on photonic and ion trap systems.

An important feature of DTQWs is that their behavior depends sensitively on the choice of coin operator. Different coins can induce localization, ballistic spread, or symmetry breaking. This tunability can be exploited in algorithm design. By contrast, CTQWs have less flexibility since the Hamiltonian is fully determined by the graph.






\section{Comparing Continuous and Discrete-Time Quantum Walks}



Continuous- and discrete-time quantum walks are closely related, yet not strictly equivalent models. Both involve the interference of quantum amplitudes on a graph and can produce similar high-level behaviors, such as linear spreading and faster hitting times compared to classical random walks. In many algorithmic contexts, either model can be used to achieve a quantum speedup.

However, several important distinctions exist between the two models:

\subsection{State Space}

\begin{itemize}
    \item \textbf{CTQW:} Evolves on the vertex Hilbert space $\mathbb{C}^{|V|}$.
    \item \textbf{DTQW (Coined):} Uses an enlarged space $\mathbb{C}^{|V|} \otimes \mathbb{C}^d$, introducing a coin register to enable unitary evolution.
\end{itemize}

The coin adds extra degrees of freedom, allowing more design flexibility. However, it also increases the complexity of analysis and implementation since one must choose both the coin and shift operators carefully to ensure desired evolution.

\subsection{Time Evolution}

\begin{itemize}
    \item \textbf{CTQW:} Evolution is governed by the continuous-time Schrödinger equation: $\ket{\psi(t)} = e^{-iHt} \ket{\psi(0)}$, allowing probing at arbitrary real-valued times $t$.
    \item \textbf{DTQW:} Evolves in discrete time-steps via a unitary operator $U$ applied iteratively: $\ket{\psi(t)} = U^t \ket{\psi(0)}$, with $t \in \mathbb{Z}$.
\end{itemize}

DTQWs can exhibit periodic revivals and step-dependent interference patterns, while CTQWs exhibit behavior determined by the spectral decomposition of $H$. A DTQW can approximate a CTQW through Trotterization: setting $U \approx e^{-iH \Delta t}$ for small $\Delta t$ and iterating.

\subsection{Generality}

\begin{itemize}
    \item \textbf{CTQW:} Any undirected graph naturally defines a Hermitian Hamiltonian and thus a valid walk.
    \item \textbf{DTQW:} For general (especially irregular) graphs, defining a local coinless unitary operator may be challenging. Enlarged Hilbert spaces or alternate frameworks like Szegedy's construction are often required.
\end{itemize}

While coined DTQWs require more setup, they are ultimately general, and Szegedy's method can handle any reversible Markov chain.

\subsection{Analytical Tools}

\begin{itemize}
    \item \textbf{CTQW:} Analysis reduces to diagonalizing the Hermitian Hamiltonian $H$.
    \item \textbf{DTQW:} One must diagonalize the larger unitary operator $U$, which may have a more intricate spectrum due to the coin space.
\end{itemize}

In both cases, analysis is often facilitated by symmetry. For example, when the underlying graph is a Cayley graph or has high symmetry, representation theory can be used to obtain explicit eigenstates and eigenvalues.

\subsection{Equivalences}

There exist formal correspondences between the two models. For instance:
\begin{itemize}
    \item A coined DTQW can approximate a CTQW in the limit of small time-steps (Childs et al.).
    \item CTQWs can be embedded into DTQWs by using additional coin degrees of freedom to emulate continuous evolution.
\end{itemize}

Thus, the two models are equivalent in computational power. However, in practice, one may be more natural or efficient depending on the problem. For example:
\begin{itemize}
    \item \textbf{CTQW:} Favored for spatial search on low-degree graphs due to the natural definition of a marked Hamiltonian.
    \item \textbf{DTQW:} Preferred for quantizing arbitrary Markov chains (e.g., element distinctness on the Johnson graph), particularly using Szegedy’s framework.
\end{itemize}

\subsection{Summary}

Continuous-time and discrete-time quantum walks are powerful and related models for quantum computation and algorithm design. CTQWs rely on continuous evolution under a Hermitian Hamiltonian and are tightly linked to physical systems. DTQWs require coin and shift operators but offer additional control and ease of implementation in gate-based architectures.

Each model has advantages:
\begin{itemize}
    \item CTQWs often more closely mirror physical intuition.
    \item DTQWs offer greater flexibility and modularity for algorithm design.
\end{itemize}

Results in one model can often be translated to the other with suitable modifications. Consequently, both are indispensable tools in the study of quantum algorithms.







\section{Cayley Graphs}



Having introduced the two types of quantum walks, we now focus on walks on highly symmetric graphs that arise from group structures. A \textbf{Cayley graph} encodes the algebraic structure of a group in graph-theoretic form and provides a natural setting for analyzing quantum walks due to its symmetry properties.

Let $G$ be a group with identity element $e$, and let $S \subset G$ be a generating set for $G$. The \emph{Cayley graph} $\Gamma(G,S)$ is defined as follows:
\begin{itemize}
    \item Each group element $g \in G$ corresponds to a vertex.
    \item For each $g \in G$ and each generator $s \in S$, there is a directed edge from $g$ to $gs$. That is, $(g, gs)$ is an edge labeled by $s$.
\end{itemize}

If $S$ is symmetric (i.e., $s \in S \Rightarrow s^{-1} \in S$) and does not contain the identity, the graph can be treated as \textbf{undirected}, since for every edge from $g$ to $gs$, there is a corresponding edge back from $gs$ to $g$.

When $S$ is symmetric, $\Gamma(G, S)$ is an undirected $|S|$-regular graph. Note that the Cayley graph depends on the choice of generating set $S$: different generating sets for the same group yield different graphs, though all share key symmetry features.

\subsection{Examples of Cayley Graphs}

\begin{itemize}
    \item \textbf{Cycle Graph $C_n$:} The cycle graph is the Cayley graph of the cyclic group $\mathbb{Z}_n = \{0, 1, \dots, n-1\}$ with generating set $S = \{1, -1\}$. Each vertex $j \in \mathbb{Z}_n$ is connected to $j \pm 1 \mod n$.

    \item \textbf{Hypercube $Q_n$:} The $n$-dimensional Boolean hypercube is the Cayley graph of the group $(\mathbb{Z}_2)^n$ with generating set $S = \{e_1, \dots, e_n\}$, where $e_i$ flips the $i$-th bit. This graph is $n$-regular and bipartite.

    \item \textbf{Complete Graph $K_N$:} The complete graph on $N$ vertices can be viewed as the Cayley graph of $\mathbb{Z}_N$ with generating set $S = \{1, 2, \dots, N-1\}$ (all nonzero elements). Each vertex $g$ has an edge to every other vertex $g+s$.

    \item \textbf{Infinite Tree (Free Group):} The free group on two generators $a$ and $b$ has a Cayley graph that is an infinite 4-regular tree. Each vertex corresponds to a word in $\{a^{\pm1}, b^{\pm1}\}$, and edges represent multiplication by $a$ or $b$. A finite portion of this graph is shown in Figure~\ref{fig:cayley-free-group}.

    \item \textbf{Dihedral Groups:} Cayley graphs of dihedral groups often form cyclic or prism-like structures. These graphs are used in modeling symmetric structures in both classical and quantum settings.
\end{itemize}

\subsection{Symmetry and Group Action}

Cayley graphs are highly symmetric. The group $G$ acts regularly on the vertices of $\Gamma(G, S)$ via left multiplication:
\[
g : h \mapsto gh.
\]
This action is:
\begin{itemize}
    \item \textbf{Vertex-transitive:} All vertices are structurally identical; there is no distinguished origin unless one is chosen explicitly.
    \item \textbf{Regular:} For any two vertices $u, v \in G$, there exists exactly one $g \in G$ such that $g \cdot u = v$.
\end{itemize}

This symmetry implies that quantum (or classical) walks can often be analyzed by considering representative initial conditions—commonly, the walk is assumed to start at the identity element $e$.

\subsection{Remarks}

Cayley graphs provide a powerful framework for analyzing quantum walks. Their regularity and symmetry often lead to analytical tractability, particularly using representation theory and spectral techniques. Throughout this thesis, we will explore quantum walks on Cayley graphs, leveraging their symmetry to obtain insights into walk dynamics and algorithmic applications.







\section{Quantum Walks on Cayley Graphs}


Cayley graphs provide an ideal setting for quantum walks due to their rich symmetry properties, which can be exploited to simplify analysis and implementation. Both continuous-time and discrete-time quantum walks on Cayley graphs have been studied extensively. We now highlight several key features and advantages of quantum walks in this setting.


%%%%%%%%%%%%%%%%%%%%%%%%%%%%%%%%%%%%%%%%%%%%%%%%%%%%%%%
\subsection{Spectral Structure}

The adjacency matrix $A$ of a Cayley graph is closely related to the group algebra of $G$. It can be expressed as:
\[
A = \sum_{a \in G} \lambda_a |\hat{a}\rangle \langle \hat{a}|,
\]
where $|\hat{a}\rangle$ is the quantum Fourier transform of $|a\rangle$. The eigenvalues $\lambda$ are given by
\[
\lambda_a = \sum_{q \in Q} \chi(a, q).
\]
Note that the eigenvectors $|\hat{a}\rangle$ of $A$ depend only on $G$ and not on the set $Q$.

To see this, we have
\[
        A\ket{\hat{a}} = A.\frac{1}{\sqrt{|G|}}\sum_{y \in G}\chi(a,y)\ket{y} = \frac{1}{\sqrt{|G|}}\sum_{y \in G}\chi(a,y).A\ket{y}
        \]
        \[
        = \frac{1}{\sqrt{|G|}}\sum_{y \in G}\chi(a,y).\sum_{q \in Q}\ket{qy}
        \]
        Consider $\beta = qy$. Then:
        \[
         = \frac{1}{\sqrt{|G|}}\sum_{q \in Q}\chi(a,q)\sum_{\beta}\chi(a,\beta)\ket{\beta} = \sum_{q \in Q}\chi(a,q) . \ket{\hat{a}}
        \]
        \[
         = \lambda_a\ket{\hat{a}}
        \]


% \[
% A = \sum_{s \in S} P_s,
% \]
% where $P_s$ is the permutation matrix representing left multiplication by $s \in G$: $P_s \ket{g} = \ket{sg}$ for all $g \in G$.

% If $S$ is symmetric (i.e., closed under inversion), then $A$ is Hermitian, and $P_s^\dagger = P_{s^{-1}}$. The eigenstates of $A$ are understood through group representation theory. In particular, the regular representation of $G$ on $\mathbb{C}^{|G|}$ decomposes into a direct sum of irreducible representations (irreps), and since each $P_s$ acts via the regular representation, the adjacency matrix $A$ is a sum of representation operators:
% \[
% A = \sum_{s \in S} \rho(s),
% \]
% where $\rho$ denotes the regular representation.

% By Schur’s lemma, $A$ acts within each irreducible subspace as a matrix proportional to the representation of $S$. Hence, $A$ is block-diagonal (in a suitable basis), with blocks corresponding to irreps of $G$. If $\{\phi_\alpha\}$ are the irreps of $G$, then the eigenvalues of $A$ can be determined from the character values $\chi_\alpha(s)$:
% \[
% \lambda_\alpha = \sum_{s \in S} \chi_\alpha(s), \quad \text{(for abelian $G$, $\chi_\alpha$ is a character)}.
% \]

% In the case of abelian $G$, all irreps are one-dimensional, and the eigenstates of $A$ are the Fourier basis:
% \[
% \ket{\chi} = \frac{1}{\sqrt{|G|}} \sum_{g \in G} \chi(g) \ket{g}, \quad \lambda_\chi = \sum_{s \in S} \chi(s).
% \]

% \textbf{Example – Hypercube:} For $G = \mathbb{Z}_2^n$, characters are labeled by bitstrings $u \in \{0,1\}^n$, and $\chi_u(g) = (-1)^{u \cdot g}$. Then
% \[
% \lambda_u = \sum_{j=1}^n (-1)^{u_j},
% \]
% which gives the $n+1$ distinct eigenvalues $n, n-2, \dots, -n$ of the $n$-dimensional hypercube.

% In the nonabelian case, similar principles apply but require using higher-dimensional irreps and the Great Orthogonality Theorem.


%%%%%%%%%%%%%%%%%%%%%%%%%%%%%%%%%%%%%%%%%%%%%%%%%%%%
\subsection{Invariant Subspaces and Symmetry Reduction}

Cayley graphs possess vertex-transitive symmetry under the group action $g : h \mapsto gh$. This symmetry induces invariant subspaces for the quantum walk evolution. If the initial state is uniform over a symmetry orbit, it remains so under the walk.

In particular, for a continuous-time quantum walk starting at the identity element:
\[
\ket{\psi(t)} = e^{-iAt} \ket{e},
\]
the resulting state remains a \emph{class function}, i.e., the amplitude $\braket{g|\psi(t)}$ depends only on structural properties of $g$ (such as Hamming weight or conjugacy class).

This symmetry drastically reduces the effective state space. For instance, on distance-regular graphs (including many Cayley graphs), amplitudes are equal for all vertices at the same distance from the starting point. This idea is also used in the analysis of the glued trees graph, where the walk remains uniform on each layer.

\subsection{Hitting and Mixing Properties}

Quantum walks on Cayley graphs have been applied to study hitting and mixing times. In the classical case, random walks on many Cayley graphs are rapidly mixing (e.g., those based on nonabelian simple groups, which often form expanders).

In the quantum case, mixing is more subtle due to unitary evolution. One typically considers \emph{instantaneous} or \emph{time-averaged} mixing. Some Cayley graphs support perfect state transfer or periodicity in CTQWs.

\textbf{Examples:}
\begin{itemize}
    \item On the cycle graph $C_n$, a CTQW exhibits perfect revival at time $t = \pi$ if $n$ is even.
    \item On the hypercube, the CTQW exhibits approximate uniform mixing at specific times.
\end{itemize}

These properties are derived from the spectral structure, which is accessible via group characters.

\subsection{Discrete-Time Walks and Symmetry}

Coined DTQWs also benefit from Cayley graph symmetry. If a symmetric coin (e.g., the Grover coin) is used at each vertex of a vertex-transitive graph, the walk respects the full symmetry. As a result:
\begin{itemize}
    \item Amplitudes often remain uniform on symmetry-equivalent vertices.
    \item Walk dynamics can often be analyzed by reducing to a lower-dimensional invariant subspace.
    \item In some cases, symmetry can lead to \emph{localization}, where amplitude becomes trapped due to spectral properties of the coin.
\end{itemize}

These effects can be mitigated or exploited through careful coin design or by using Szegedy's method, which naturally respects symmetry.

\subsection{Summary}

Cayley graphs offer a rich and powerful structure for quantum walks. Their group-theoretic properties allow:
\begin{itemize}
    \item Spectral analysis via irreducible representations.
    \item Dimensionality reduction using symmetry and invariant subspaces.
    \item Explicit calculation of probabilities, mixing behavior, and hitting times.
\end{itemize}

Many efficient quantum walk algorithms exploit these properties either directly or implicitly. In subsequent sections, we will explore quantum walks on more general structures where group actions go beyond simple Cayley graphs.







\section{Cayley Graphs with Group Action}


Cayley graphs can also be generalized through group actions. Let \( (G, X, *) \) be a regular group action with a fixed element \( x \in X \), and let \( Q = \{q_1, q_2, \dots, q_k\} \subset G \) be a subset as before. Define the Cayley graph \( \Gamma = (X, E) \), where the vertex set is \( X \), and the edge set consists of all pairs \( (x, y) \in X \times X \) such that \( y = q * x \) for some \( q \in Q \).

The adjacency matrix of this graph can be written as:
\[
A = \sum_{h \in G} \lambda_h \ket{G^{(h)} * x} \bra{G^{(h)} * x},
\]
where
\[
\lambda_h = \sum_{q \in Q} \chi(h, q), \quad \text{and} \quad \ket{G^{(h)} * x} = \frac{1}{\sqrt{|G|}} \sum_{g \in G} \chi(g, h) \ket{g * x}.
\]

As in the previous case, the eigenvectors \( \ket{G^{(h)} * x} \) are determined solely by the structure of the group \( G \). This construction generalizes the standard Cayley graph: setting \( X = G \) and the group action \( * \) as group multiplication recovers the original definition.


To see this, we have:

\[
    A\ket{G^{(h)}*x} = \frac{1}{\sqrt{|G|}}.\sum_{g \in G}\chi(g,h) A.\ket{g*x}
    \]

    \[
    = A\ket{G^{(h)}*x} = \frac{1}{\sqrt{|G|}}.\sum_{g \in G}\chi(g,h)\sum_{q \in Q}\ket{q*(g*x)}
    \]
    Consider $\beta = qg$. Then:
    \[
    = = \frac{1}{\sqrt{|G|}}\sum_{q \in Q}\sum_{\beta \in G}\chi(\beta,h)\chi(q,h)\ket{\beta * x}
    \]

    \[
    = \sum_{q \in Q}\chi(q,h).\ket{G^{(h)}*x} = \lambda_h \ket{G^{(h)}*x}
    \]

    Since the group action \( (G, X, *) \) is regular, the two Cayley graph constructions are isomorphic. In the first construction, the vertex set is \( G \), and the adjacency matrix is indexed by elements of \( G \). In the second construction, the vertex set is \( X \), with the adjacency matrix indexed by elements of \( X \). The isomorphism between these graphs is given by the bijection
\[
\setlength{\arraycolsep}{1mm}
\begin{array}{rcll}
\phi: & G & \longrightarrow & X \\
      & g & \longmapsto     & g * x,
\end{array}
\]

which maps each group element \( g \in G \) to the corresponding point in its orbit under the action.


    % Assuming the group action is a cryptographic group action, the isomorphism $\phi$ is a one-way function: given $g \in G$, it is easy to compute $g * x$, whereas given $(x, g * x)$, it is hard to compute $g$. Therefore, although these graphs are mathematically the same, computational problems based on them require fundamentally different techniques. A notable example is the implementation of quantum walks on these graphs.\footnote{Implementations of continuous-time quantum walks in the context of cryptographic group actions have previously appeared in \cite{booher2024failing, doliskani2023sample}, in the setting of supersingular isogeny graphs. These walks can be interpreted as group-action walks.} From this point onward, assume $Q$ is a small set, i.e., $\abs{Q} = \poly(\log \abs{G})$. In the first graph, we can efficiently implement the walks $e^{-iAt}$ even for values of $t$ that are exponentially large in $\log \abs{G}$. This follows from the fact that
    % \begin{equation}
    %     \label{eq:adjmat-fb}
    %     e^{-iAt} = \qft_G \sum_{a \in G} e^{-i\lambda_a t} \ket{a} \bra{a} \qft_G^*.
    % \end{equation}
    % The quantum Fourier transform $\qft_G$ and its inverse can be applied in $\poly(\log \abs{G})$ operations. The diagonal unitary in the middle, which is a phase computation $\ket{a} \mapsto e^{-i\lambda_a t} \ket{a}$, is also efficient because $\lambda_a$ can be computed classically to arbitrary precision in $\poly(\log \abs{G})$ time.
    
    % The situation for group actions is less straightforward. While the states $\ket{G^{(h)} * x}$ are analogous to the states $\ket{\hat{h}}$, the involvement of the action $*$ in the former introduces computational challenges. Specifically, it becomes hard to apply transformations beyond the action $\ket{y} \mapsto \ket{a * y}$ for $a \in G$. Consequently, no decomposition analogous to \eqref{eq:adjmat-fb} exists in the context of group actions. Instead, we are left with the spectral expression
    % \[ e^{-iAt} = \sum_{h \in G} e^{-i\lambda_h t} \ket{G^{(h)} * x} \bra{G^{(h)} * x}. \]
    % Despite this limitation, the sparsity and structure of the matrix $A$ allow us to demonstrate in the next section that for $t = \poly(\log \abs{G})$, the walk $e^{-iAt}$ can still be efficiently approximated to polynomial accuracy.   


    When the group action \( (G, X, *) \) is assumed to be cryptographic, the isomorphism \( \phi \) becomes a one-way function: given \( g \in G \), it is efficient to compute \( g * x \), but given the pair \( (x, g * x) \), it is computationally hard to recover \( g \). As a result, while the two Cayley graphs are mathematically isomorphic, computational tasks associated with them may require fundamentally different approaches. A particularly illustrative example is the implementation of quantum walks on these graphs.\footnote{Continuous-time quantum walks have been explored in the context of cryptographic group actions in~\cite{booher2024failing, doliskani2023sample}, specifically in supersingular isogeny graphs. These walks can be viewed as group-action quantum walks.}



% The concept of a Cayley graph generalizes naturally to the case of a group acting on an arbitrary set. Suppose a group $G$ acts transitively on a set $X$, and let $S \subset G$ be a generating set. We can construct a graph, often referred to as a \textbf{Schreier graph}, denoted $\Gamma(X; G, S)$, as follows:
% \begin{itemize}
%     \item The vertex set is $X$.
%     \item For each $x \in X$ and each $s \in S$, there is an edge from $x$ to $s \cdot x$.
% \end{itemize}

% This graph encodes the action of $G$ on $X$ and generalizes the Cayley graph construction. In fact, when $X \cong G/H$ for some stabilizer subgroup $H$ of a basepoint $x_0$, we interpret $\Gamma(X; G, S)$ as the \emph{Cayley graph of $G$ acting on the coset space $G/H$}. If $H$ is trivial (i.e., the action is regular), then $X \cong G$ and $\Gamma(X; G, S)$ reduces to the standard Cayley graph $\Gamma(G, S)$.

% When $H$ is nontrivial, several group elements from $G$ may correspond to the same vertex in $X$. Nonetheless, $\Gamma(X; G, S)$ inherits considerable symmetry from the group action:
% \begin{itemize}
%     \item $G$ acts by automorphisms on the graph,
%     \item The local structure of the graph is uniform across orbits.
% \end{itemize}

% \subsection{Examples}

% \begin{itemize}
%     \item \textbf{Symmetric group:} Let $G = S_n$ act on $X = \{1, \dots, n\}$ by permutations, and let $S$ be a generating set of transpositions. Then $\Gamma(X; G, S)$ is a complete graph on $n$ vertices, as any element can be swapped with any other via some $s \in S$.

%     \item \textbf{Johnson graphs:} These are Schreier graphs of $S_n$ acting on $k$-element subsets of $\{1, \dots, n\}$. Generators correspond to transpositions that change one element of the subset. Johnson graphs are highly symmetric and distance-regular, and they appear frequently in quantum walk algorithms.
% \end{itemize}

% \subsection{Quantum Walks on Group Action Graphs}

% Quantum walks on graphs arising from group actions share many of the useful features of walks on full Cayley graphs. In fact, one can often analyze a walk on $\Gamma(X; G, S)$ by \emph{lifting} it to a walk on the Cayley graph $\Gamma(G, S)$ and projecting down via the quotient map $G \to G/H$. This structure preserves symmetry in the transition amplitudes.

% In particular, if the initial quantum state is uniform over an orbit of $H$, the state will remain symmetric under the subgroup $H$ throughout the evolution. Thus, the walk evolves in an \emph{invariant subspace}, reducing the effective size of the problem.

% \textbf{Example – Quotient Walks:} Krovi and Brun (2007) analyzed quantum walks on quotient graphs and showed how hitting probabilities and mixing behavior can be improved by exploiting such symmetry reductions.

% Group action graphs are particularly useful in algorithmic settings where the state space is naturally modeled as an orbit under a group of moves or transformations. For instance:
% \begin{itemize}
%     \item The Rubik's cube configuration space is essentially the Cayley graph of the Rubik’s group acting on itself. Each configuration corresponds to a group element.
%     \item A quantum walk on this space is a specialized form of group action walk, where subgroups preserving partial configurations can be used to reduce search complexity.
% \end{itemize}

% \subsection{Summary}

% Graphs defined from group actions—such as Schreier graphs $\Gamma(X; G, S)$—generalize Cayley graphs to settings where the vertex set is a homogeneous space $G/H$. These graphs:
% \begin{itemize}
%     \item Maintain significant symmetry,
%     \item Enable reduction to invariant subspaces via subgroup structure,
%     \item Retain many spectral and dynamical properties of Cayley graphs.
% \end{itemize}

% Quantum walks on such graphs behave similarly to those on full Cayley graphs, though with potentially fewer vertices due to coset identification. This perspective is powerful for analyzing and implementing quantum algorithms.

In the next section, we develop techniques for simulating continuous-time quantum walks on such symmetric graphs using quantum circuits. These simulations exploit the group action structure to simplify circuit construction and enhance algorithmic efficiency.




\chapter{Simulating Group Action Quantum Walks}



From this point forward, we assume the generating set \( Q \) is small, i.e., \( \abs{Q} = \operatorname{poly}(\log \abs{G}) \). In the first construction (where \( X = G \)), the walk \( e^{-iAt} \) can be efficiently implemented even for values of \( t \) that are exponentially large in \( \log \abs{G} \). This is a consequence of the identity:
\begin{equation}
    \label{eq:adjmat-fb}
    e^{-iAt} = \operatorname{QFT}_G \left( \sum_{a \in G} e^{-i\lambda_a t} \ket{a} \bra{a} \right) \operatorname{QFT}_G^*,
\end{equation}
where \( \operatorname{QFT}_G \) and its inverse can be implemented in \( \operatorname{poly}(\log \abs{G}) \) time. Moreover, the diagonal phase unitary \( \ket{a} \mapsto e^{-i\lambda_a t} \ket{a} \) is also efficient to implement, since the eigenvalues \( \lambda_a \) can be computed classically to arbitrary precision in \( \operatorname{poly}(\log \abs{G}) \) time.

In contrast, the group-action setting presents greater computational difficulty. Although the states \( \ket{G^{(h)} * x} \) play a role analogous to the Fourier basis states \( \ket{\hat{h}} \), the presence of the action \( * \) makes the situation more subtle. In particular, implementing general transformations over these states is difficult; we are typically limited to operations of the form \( \ket{y} \mapsto \ket{a * y} \) for \( a \in G \). As a result, no efficient decomposition similar to~\eqref{eq:adjmat-fb} is known in the group-action setting. Instead, the best we can express is the spectral decomposition:
\[
e^{-iAt} = \sum_{h \in G} e^{-i\lambda_h t} \ket{G^{(h)} * x} \bra{G^{(h)} * x}.
\]

Despite this limitation, the adjacency matrix \( A \) retains a high degree of sparsity and algebraic structure. In the next section, we will show that for \( t = \operatorname{poly}(\log \abs{G}) \), the walk \( e^{-iAt} \) can still be efficiently approximated to polynomial accuracy.

%%%%%%%%%%%%%%%%%%%%%%%%%%%%%%%%%%%%%%%%%%%%%%%%%%%%%%%


\subsection{Simulating continuous-time walks}
Childs and Berry, Childs, and Kothari showed that continuous-time quantum walks can be simulated using discrete-time quantum walks. Since our approach relies on their methods, we begin by outlining the relevant notation and framework.

Let \( H \) be a Hamiltonian of dimension \( N = 2^n \), and define \( \opnorm{H}_{\mathrm{max}} = \max_{i, j} \abs{H_{ij}} \). To facilitate the simulation, we expand the Hilbert space from \( \mathbb{C}^N \) to \( \mathbb{C}^{2N} \otimes \mathbb{C}^{2N} \) by appending an ancilla qubit initialized to \( \ket{0} \) and duplicating the resulting space.

To define the discrete-time walk operator, we first construct an orthonormal set of states:
\begin{align}
    \ket{\phi_{j0}} & := \frac{1}{\sqrt{d}} \sum_{\ell \in F_j} \ket{\ell} \left( \sqrt{\frac{H_{j\ell}^*}{K}} \ket{0} + \sqrt{1 - \frac{\abs{H_{j\ell}^*}}{K}} \ket{1} \right), \label{eq:walk-sim} \\
    \ket{\phi_{j1}} & := \ket{0} \ket{1}, \nonumber
\end{align}
where \( F_j \) denotes the set of nonzero entries in column \( j \) of \( H \), and \( K \ge \opnorm{H}_{\mathrm{max}} \) is a fixed constant.

Based on these states, we define the isometry \( T : \mathbb{C}^{2N} \to \mathbb{C}^{2N} \otimes \mathbb{C}^{2N} \) as follows:
\begin{equation}
    \label{eq:walk-isom}
    T := \sum_{j = 0}^{N - 1} \sum_{b \in \{0, 1\}} (\ket{j}\bra{j} \otimes \ket{b}\bra{b}) \otimes \ket{\phi_{jb}}.
\end{equation}

The discrete-time quantum walk operator is then defined as
\[
W = iS(2TT^* - \mathbf{1}),
\]
where \( S \) is the swap operator acting as
\[
S \ket{j_1} \ket{b_1} \ket{j_2} \ket{b_2} = \ket{j_2} \ket{b_2} \ket{j_1} \ket{b_1},
\]
for all \( 0 \le j_1, j_2 < N \) and \( b_1, b_2 \in \{0, 1\} \).

To efficiently simulate the continuous-time evolution \( e^{-iHt} \), it suffices to implement the isometry \( T \), its adjoint \( T^* \), and the walk operator \( W \) efficiently.
%  Assuming black-box (oracle) access to the entries of \( H \), the following result holds:

% \begin{theorem}[{\cite[Theorem 1]{berry2015hamiltonian}}]
% \label{thm:sparse-sim}
% Let \( H \) be a \( d \)-sparse Hamiltonian acting on \( n \) qubits. Then the unitary \( e^{-iHt} \) can be approximated to within error \( \epsilon \) using
% \[
% O\left( \tau \frac{\log(\tau / \epsilon)}{\log\log(\tau / \epsilon)} \right)
% \]
% queries to \( H \), and
% \[
% O\left( \tau \left[ n + \log^{5 / 2}(\tau / \epsilon) \right] \frac{\log(\tau / \epsilon)}{\log\log(\tau / \epsilon)} \right)
% \]
% additional elementary gates, where \( \tau = d \cdot \opnorm{H}_{\mathrm{max}} \cdot t \).
% \end{theorem}







\subsection{Simulating group action quantum walks.}
Assume \( t = \operatorname{poly}(\log \abs{G}) \). We show that the continuous-time walk \( W = e^{-iAt} \) can be efficiently simulated using the discrete-time quantum walk framework developed in~\cite{childs2010relationship, berry2015hamiltonian}. Recall the isometry \( T \) defined in Equation~\eqref{eq:walk-isom}, where the states \( \ket{\phi_b} \) are as given in Equation~\eqref{eq:walk-sim}. To carry out the simulation, it suffices to show that the isometry \( T \), its adjoint \( T^* \), and the discrete-time walk operator
\[
W = iS(2TT^* - \mathbf{1})
\]
can all be efficiently implemented in the group-action setting.


Let \( \Gamma = (X, E) \) be the Cayley graph associated with the group action \( (G, X, *) \), and let \( A \) denote its adjacency matrix. The Hamiltonian \( H \) of interest is simply \( A \), indexed by elements of \( X \). In this context, the isometry \( T \) takes the form:
\begin{align*}
T
& := \sum_{y \in X} \sum_{b \in \{0, 1\}} (\ket{y}\bra{y} \otimes \ket{b}\bra{b}) \otimes \ket{\phi_{yb}} \\
& = \sum_{y \in X} \sum_{b \in \{0, 1\}} \ket{y} \ket{b} \ket{\phi_{yb}} \bra{y} \bra{b}.
\end{align*}


Because of the structure of $\Gamma$, the states .... simplify as follows. First, since the set $Q$ is symmetric, the graph $\Gamma$ is undirected and the adjacency matrix $A$ is symmetric with non-negative entries; the nonzero entries of $A$ are all equal to $1$. Therefore, we can set $K = {A}_{\mathrm{max}} = 1$. As a result, for $y \in X$, the state $\ket{\phi_{y0}}$ becomes
\[ \ket{\phi_{y0}} = \frac{1}{\sqrt{\abs{Q}}} \sum_{q \in Q} \ket{q * y} \ket{0}. \]
We now describe how to implement the isometry \( T \) by constructing efficient unitaries \( U_0 \) and \( U_1 \), where each \( U_b \) acts as
\[
U_b : \ket{0} \ket{y, b} \ket{0, 0} \mapsto \ket{0} \ket{y, b} \ket{\phi_{yb}}, \quad b \in \{0, 1\},
\]
with the first register serving as an ancilla.

The unitary \( U_1 \) is straightforward to implement, since \( \ket{\phi_{y1}} = \ket{0} \ket{1} \) is independent of \( y \). We now focus on the construction of \( U_0 \).

Because \( Q \subset G \) is a small set (i.e., \( \abs{Q} = \operatorname{poly}(\log \abs{G}) \)), we can construct an efficient unitary \( V_Q : \mathbb{C}^Q \to \mathbb{C}^Q \) such that
\[
V_Q \ket{0} = \frac{1}{\sqrt{\abs{Q}}} \sum_{q \in Q} \ket{q}.
\]
Applying \( V_Q \otimes \mathbf{1} \) to the state \( \ket{0} \ket{y, 0} \ket{0, 0} \) produces:
\[
\frac{1}{\sqrt{\abs{Q}}} \sum_{q \in Q} \ket{q} \ket{y, 0} \ket{0, 0}.
\]

Next, we apply the unitary \( V_1 \), defined by:
\[
V_1 : \ket{q} \ket{y, 0} \ket{0, 0} \mapsto \ket{q} \ket{y, 0} \ket{q * y, 0}.
\]

Following this, we uncompute the first register to recover \( \ket{0} \ket{y, 0} \ket{\phi_{y0}} \). This is achieved by applying the unitary \( V_2 \), defined as:
\[
V_2 : \ket{q} \ket{y, 0} \ket{q * y, 0} \mapsto \ket{0} \ket{y, 0} \ket{q * y, 0}.
\]
Since \( Q \) is small, we can efficiently determine \( q \) from the pair \( (y, q * y) \) by exhaustively checking each \( q \in Q \). Thus, \( V_2 \) (and therefore \( U_0 \)) can be implemented efficiently.


To implement \( U_0^* : \ket{0} \ket{y, 0} \ket{\phi_{y0}} \mapsto \ket{0} \ket{y, 0} \ket{0, 0} \), we sequentially apply \( V_2^* \), followed by \( V_1^* \), and finally \( V_Q^* \otimes \mathbf{1} \). Since each of these unitaries can be implemented efficiently, it follows that \( U_0^* \) is also efficient.

We now construct the isometry \( T \) using a conditional unitary \( U_T \), which applies either \( U_0 \) or \( U_1 \) depending on the value of the qubit \( b \). That is,
\[
U_T = \text{Controlled-}U_b \quad \text{with control on } b \in \{0,1\}.
\]

Next, we demonstrate that the walk unitary
\[
W = iS(2TT^* - \mathbf{1})
\]
can be applied efficiently. This reduces to implementing the reflection \( 2TT^* - \mathbf{1} \), which we analyze as follows:
\begin{align*}
2\ket{0}\bra{0} \otimes TT^* - \mathbf{1}
&= 2\ket{0}\bra{0} \otimes \sum_{y \in X} \sum_{b \in \{0,1\}} \ket{y,b} \ket{\phi_{yb}} \bra{y,b} \bra{\phi_{yb}} - \mathbf{1} \\
&= 2\sum_{y \in X} \sum_{b \in \{0,1\}} \ket{0} \ket{y,b} \ket{\phi_{yb}} \bra{0} \bra{y,b} \bra{\phi_{yb}} - \mathbf{1} \\
&= U_T \left( 2 \sum_{y \in X} \sum_{b \in \{0,1\}} \ket{0} \ket{y,b} \ket{0,0} \bra{0} \bra{y,b} \bra{0,0} - \mathbf{1} \right) U_T^* \\
&= U_T \left( 2\ket{0}\bra{0} \otimes \mathbf{1}_{X,b} \otimes \ket{0,0}\bra{0,0} - \mathbf{1} \right) U_T^*.
\end{align*}

Since \( U_T \), \( U_T^* \), and the operator \( 2\ket{0}\bra{0} \otimes \mathbf{1}_{X,b} \otimes \ket{0,0}\bra{0,0} - \mathbf{1} \) can all be implemented efficiently, it follows that the reflection \( 2\ket{0}\bra{0} \otimes TT^* - \mathbf{1} \) is also efficient.

Finally, for any input state \( \ket{\psi} \), the reflection acts as:
\[
\left(2\ket{0}\bra{0} \otimes TT^* - \mathbf{1} \right) \ket{0} \ket{\psi} = \ket{0} \left(2TT^* - \mathbf{1} \right) \ket{\psi},
\]
confirming that the full walk operator \( W \) can be implemented efficiently in the group-action setting.





\section{Hamiltonian Structure and Sparsity}

\section{Quantum Walk-Based and QSP Simulation}

\section{Complexity Analysis}




\chapter{Quantum Money and Hartley Transform}





\chapter{Verification Algorithm Using Quantum Walks}





\bibliographystyle{plain}
\begin{thebibliography}{99}
% Sample Bib entries. Replace with real ones.
\bibitem{Aaronson2009} S. Aaronson, ``Quantum Copy-Protection and Quantum Money,'' In Proc. of CCC, 2009.
\bibitem{Zhandry2022} M. Zhandry, ``Quantum Money from Modular Forms,'' In Proc. of CRYPTO, 2022.
\bibitem{Childs2003} A. M. Childs, ``Quantum Information Processing in Continuous Time,'' PhD Thesis, MIT, 2004.
% Add all other references as needed
\end{thebibliography}




\end{document}
